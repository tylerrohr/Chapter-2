%%%%%%%%%%%%%%%%%%%%%%%%%%%%%%%%%%%%%%%%%%%%%%%%
%%      3.2 Eddy induced variability in bottom-up biogeochemical controls %%
%%%%%%%%%%%%%%%%%%%%%%%%%%%%%%%%%%%%%%%%%%%%%%%%%

\subsection{Eddy induced variability in bottom-up biogeochemical controls}

To understand how eddies modify the simulated distribution of Southern Ocean biomass (see \textbf{Fig. 2e-f}) it is first critical to understand how they modify community mean division rates ($\mu_\Sigma$). Variability in $\mu_\Sigma$ is a direct reflection of bottom-up growth condition and is primarily driven by light \parencite{Fauchereauresponsephytoplanktonbiomass2011} and iron \parencite{BoydEnvironmentalFactorsControlling2002} in the Southern Ocean. The pathways by which eddies modify these biogeochemical controls are depicted by the schematic in \textbf{Fig. 1}. Here we examine regional and seasonal variability in the anomalous contribution of light (\textbf{Sec. 3.2.1, Fig. 4}) and iron (\textbf{Sec. 3.2.2, Fig. 5}) to the depth integrated phytoplankton populations within cyclones and anticyclones, to better understand the mechanisms that lead to anomalous division rates (\textbf{Sec. 3.2.3, Fig.  6}). 


%%%%%%%%%%%%%%%
%% 3.2.1 MLD %%
%%%%%%%%%%%%%%%
\subsubsection{Mixed Layer Depth | $MLD$}

When considering the entire set of identified eddies, cyclones and anticyclones are essentially equally likely to exhibit anomalously deep mixed layers ($+ MLD'$; 50\% of cyclones, 51\% of anticyclones) or anomalously shallow mixed layers ($- MLD'$; 50\% of cyclones, 49\% of anticyclones). Eddies that behave in the theoretically expected direction (-(+)$MLD'$ in cyclones (anticyclones)), however, exhibit more intense deformation. Cyclones (anticyclones) with -(+) $MLD'$ produce an anomaly with an average magnitude that is roughly 60\% (60\%) larger than anticyclones (cyclones) with -(+)$MLD'$ (see \textbf{Tab. 1}).

Further, the distribution of $MLD'$ is not uniform in time or space. Seasonally, the average magnitude of $MLD'$ is 12m (or $\sim8$ times) larger during the deep mixing winter months (i.e. J,A,S) relative to the shallow mixing summer months (i.e. J,F,M) - (\textbf{Fig. 4}). This dichotomy holds even if $MLD'$ is normalized by the climatologic mixing depth. On average, summer eddies deform the $MLD$ by roughly 4\% of the climatology while winter eddies lead to 12\% deformation. Geographically, the dominant winter signal is along the Antarctic Circumpolar Current (ACC - defined between $-20cm$ and $-80cm$ in $\overline{SSH}_{Clim}$), where 78\% (75\%) of cyclones (anticyclones) exhibit $-(+)MLD'$  (\textbf{Fig. 4b, c}) with a mean of $-18m (16m)$. South of the ACC, particularly in ice covered water and the south Atlantic the direction is generally flipped. Several physical factors appear to be driving this variability.

The first is eddy size. As both eddy radius (\textbf{Fig. 7a}) and eddy amplitude (\textbf{Fig. 7b}) increase, cyclones (anticyclones) exhibit $-(+) MLD'$ with increasing magnitude and frequency (see \textbf{Tab. 1}). When considering only the subset of larger eddies ($L_s>50km, \  Amp.>10cm$) 60\% (60\%) of cyclones (anticyclones) exhibit $-(+) MLD'$  with a mean of $-13m \ (13m)$. Of the remaining 40\% (40\%) of cyclones (anticyclones) with $+(-) MLD'$, the mean magnitude is only 
$3m \ (-4m)$. This trend continues as eddies increase in size (\textbf{Fig. 7a, b})

The second physical factor is the background mixing climatology ($\overline{MLD}_{clim}$). When considering only eddies with deep background mixing ($\overline{MLD}_{clim}>100m$), 56\% (67\%) of deep mixing cyclones (anticyclones) have a $-(+) MLD'$ with a mean of $-23m \ (22m)$. Of the remaining 44\% (33\%) of cyclones (anticyclones) with $+(-) MLD'$, the mean magnitude is only $14m \ (-10m)$. The impact of background mixing on $MLD'$ is clearly reflected in the geographic and seasonal structure of the $MLD'$ distribution (\textbf{Fig. 4}). 

Together, it is predominately small eddies passing through regions with otherwise weak $\overline{MLD}_{clim}$ that exhibit behavior that is contrary to theory. Filtered for only larger eddies ($L_s>50km, \ Amp.>10cm$) with deep background mixing ($\overline{MLD}_{clim}>100m$) ($\sim$11,000 eddy realizations), 80\% (82\%) of cyclones (anticyclones) exhibit $-(+)MLD'$ with a mean of $-51m \ (57m)$. See \textbf{Tab. 1} for additional statistics on different combinations of eddy subsets.

Finally, cyclones (anticyclones) beneath heavy sea ice coverage ($>80$\% fractional ice coverage)  preferentially exhibit deeper (shallower) mixed layer. 75\% (77\%) of cyclones (anticyclones) below heavy sea exhibit $+(-)MLD'$. 



%%%%%%%%%%%%%%%%%%%%%%%%%
%% 3.2.2 Dissolved Iron %% 
%%%%%%%%%%%%%%%%%%%%%%%%%
\subsubsection{Dissolved Iron |  $[Fe]_\Sigma'$}

Averaged across all eddies, the community mean dissolved iron concentration $[Fe]_\Sigma$ is deflated ($-[Fe]_\Sigma'$) in 73\% of cyclones and elevated ($+[Fe]'_\Sigma$) in 71\% anticyclones (see \textbf{Fig. 5}) with a mean magnitude of .014 $\mu mol/m^3$. When normalized by co-located climatologic values ($[Fe]''_\Sigma$), the average eddy modifies background iron by 10.4 \%.

Geographically (\textbf{Fig. 5}), the distribution is further skewed south of the northernmost edge of the ACC ($SSH<-80cm$), with 78\% (76\%) of cyclones (anticyclones) exhibiting $-(+)[Fe]'_\Sigma$. The most notable exception is in the "Northern" South Pacific, north of the ACC ($SSH>-80cm$) and between 90W and 150W, where only 51\% (56\%) of cyclones (anticyclones) exhibit $-(+)[Fe]'_\Sigma$, albeit with weaker mean magnitudes ($abs([Fe]'_\Sigma) = .006 \ \mu mol/m^3$; $abs([Fe]''_\Sigma) = 6.7\%$). Seasonal variability is substantially weaker than seen in $MLD'$ (see \textbf{Fig. 4}). The magnitude of $[Fe]'_\Sigma$ is on average 48\% larger during the winter (JAS) relative to the summer (JFM), but only 5\% larger when considering normalized values ($[Fe]''_\Sigma$). Like $MLD'$, the magnitude of $[Fe]'_\Sigma$ increases with eddy size and the climatologic mixing depth (\textbf{Fig. 7}). Unlike $MLD'$, however, heavy ice coverage does not flip the direction of the expected anomaly. 

Mechanistically, several pathways contribute to the likely depression (elevation) of $[Fe]_\Sigma$  in cyclones (anticyclones). First, as $MLD$ is modified, so too is access to deep iron-rich waters. In turn, the upward vertical mixing flux ($Mix_{Fe}, \ \mu mol/m^3/5 days$) is typically depressed in eddies with $-MLD'$ and elevated in eddies with $+MLD'$ (\textbf{Fig. 7c-d}). The magnitude of $Mix_{Fe}$ increases both as the depth of climatologic mixing increases (\textbf{Fig. 7a, b}) and as the magnitude of the mixing anomaly increases (\textbf{Fig. 7c, d}). Vertical mixing statistically supplies less (more) iron to phytoplankton in cyclones (anticyclones). The community mean mixing tendency ($\frac{[Fe]}{dt}_{\Sigma, M}$, see \textfbf{Sec. 2.3}), is depressed in 72 \% of cyclones and elevated in 68 \% of anticyclones (see Supplemental \textbf{Fig. S1}). Note, however, that cyclones (anticyclones) with $+(-)MLD'$ do not consistently experience $+(-)Mix'_{Fe}$ as would be expected if mixing was the single dominant transport mechanism altering the iron profile (\textbf{Fig. 7c, d}). The magnitude of $Mix'_{Fe}$ is much larger in shoaling cyclones and deepening anticyclones relative to shoaling anticyclones or deepening cyclones. This suggests that mixing is not the only, nor dominant, iron supply pathway modified by eddies. 

A second, polarity dependent, pathway is Ekman pumping. The upward vertical advection of iron ($W_{Fe},\ \mu mol/ m^3/5 days$) is predominately depressed in cyclones and elevated in anticyclones (\textbf{Fig. 7a-d}). In turn, the community mean vertical advection tendency ($\frac{[dFe]}{dt}_{\Sigma, W}$), is depressed in 74\% of cyclones and elevated in 75\% of anticyclones (see Supplemental \textbf{Fig. S2}). 

Two physical mechanisms could theoretically drive the anomalous vertical advection of iron in eddies. The first, eddy pumping, is unlikely to dominate, otherwise we would expect upwelling (downwelling) in cyclones (anticyclones) that is isolated to a transient period during formation. We, however, predominately see downwelling (upwelling) in cyclones (anticyclones) that is maintained throughout the lifetime of eddies (\textbf{Fig. 7e, f}). A small contribution from eddy pumping may, however, explain the slight depression in the magnitude of $W_{Fe}'$ apparent over the first month of eddy formation. 

Ekman Pumping instead appears to be the dominate control on the anomalous vertical advection of iron in eddies. Anomalous vertical velocities induced by the rotating surface currents ($V'_{Ek}$) are negative (downward) in 89\% of cyclones and positive (upward) in 88\% of anticyclones (see Supplemental \textbf{Fig. S3}). As Ekman induced vertical advection increases (or decreases), so too does the flux of iron via vertical advection ($W_{Fe}$, \textbf{Fig. 7g, h}. Across all eddies, $\frac{[dFe]}{dt}'_{\Sigma, W}$ is positively correlated with $V'_{Ek}$ (r =.49). When only considering the middle 50\% of $\frac{[dFe]}{dt}'_{\Sigma, W}$ values the strength of this correlation increases to $r=.74$. 

Horizontally, eddies can also stir or trap iron across lateral gradients. 31 \% of all eddies with substantially modified (> 50\% percentile) vertical iron transport ($abs(W'_{Fe}+Mix'_{Fe})$) experience $[Fe]'_\Sigma$ in the opposite direction of the anomalous vertical supply of iron, implying a potentially competing role of lateral advection. In the Southern Ocean, $-(+)SSH'$ is more likelihood to emerge on the more(less) dense side of fronts \parencite{SongSeasonalvariationcorrelation2018}. Combined with the spatial distribution of iron gradients, this leads to the increased likelihood for cyclones (anticyclones), particularly meanders, to protrude up the iron gradient and produce a -(+) anomaly. (NEED TO CLARIFY THIS I THINK). The $Adv. \ Potential$  is negative in 64\% of cyclones and positive 58 \% of anticyclones.  

Finally, the community mean iron biological tendency ($\frac{[dFe]}{dt}_{\Sigma, J}$), which accounts for the iron source/sink via biological uptake (sink) and remineralization (source), is inflated (less net biological consumption) in 66\% of cyclones and elevated (more net biological consumption) in 61\% of anticyclones.

Please see (\textbf{Sec. 4.2} for a detailed discussion of the relative contribution from each mechanism.  




%%%%%%%%%%%%%%%%%%%%%%%%%%%
%$  3.2.3 Division Rates %%
%%%%%%%%%%%%%%%%%%%%%%%%%%%
\subsubsection{Population Specific Division Rates | $\mu_\Sigma$ }

Community mean population specific division rates ($\mu_\Sigma$), a direct reflection of bottom-up growth conditions, are widely deflated in cyclones and inflated in anticyclones. 80\% (77\%) of cyclones (anticyclones) exhibit ($-(+)\mu'_\Sigma$) with an average magnitude of $\sim.008$, or roughly 6.5\% when normalized by climatologic values ($\mu''_\Sigma$). 

Seasonally, the distribution is more skewed in the summer, when 85\% (84\%) of cyclones (anticyclones) exhibit $-(+)\mu_\Sigma'$, relative to the winter, when only 67\% (66\%) of cyclones (anticyclones) exhibit $-(+)\mu_\Sigma'$ (see \textbf{Fig. 6}). Geographically, this varaibility can be accounted for by an intrusion of $+(-)\mu_\Sigma'$ in cyclones (anticyclones) along the ACC during the deep mixing winter months. 

Variability in $\mu_\Sigma$ is primarily driven by light \parencite{Fauchereauresponsephytoplanktonbiomass2011} and iron \parencite{BoydEnvironmentalFactorsControlling2002} in the Southern Ocean, both of which are directly modified by eddy processes. The temporal and spatial distribution of light limitation ($(L_\Sigma^{I_{PAR}})'$) and iron limitation ($(L_\Sigma^{Fe}')$) map well onto the distribution $MLD'$ and $[Fe]'_\Sigma$ (see Supplemental \textbf{Fig. S4, S5}).

Community mean light limitation anomalies $(L_\Sigma^{I_{PAR}}')$ generally decrease (more stress) as $MLD'$ increases. There is a negative correlation between $L_\Sigma^{I_{PAR}}'$ and $MLD'$ in both cyclones and anticyclones ($r=-.55$ and $r=-.53$, respectively). The strength of this correlation jumps to $r=-.89$ ($r=-.80$) in large($L_s>50km, Amp.>10cm$), deep mixing ($\overline{MLD}_{clim}>100m$) cyclones (anticyclones). Overall, 54\% (51\%) of cyclones (anticyclones) exhibit reduced (exacerbated) light limitation ($+(-)L_\Sigma^{I_{PAR}}'$) with an average magnitude of of .008 (.008), or 4.5\% (4.3\%) when normalized by climatologic values ($L_\Sigma^{I_{PAR}}''$ ). When considering only large, deep mixing eddies, 88\% (90\%) of cyclones (anticyclones) exhibit $+(-)L_\Sigma^{I_{PAR}}'$ with an average magnitude of .017 (.016), or 11.9\% (11.2\%) when normalized by climatologic values.

Community mean iron limitation anomalies ($L_\Sigma^{Fe}'$) generally increase (less stress) as $[Fe]'_\Sigma$ increases. There is a positive correlation between $[Fe]'_\Sigma$ and $L_\Sigma^{Fe}'$ in both cyclones and anticyclones ($r=.60$ and $r=.58$, respectively). The strength of this correlation jumps to $r=.81$ ($r=.73$) in cyclones (anticyclones) when only considering large, deep mixing eddies (same as above). Overall, 74\% (72\%) of cyclones (anticyclones) exhibit exacerbated (reduced) iron limitation ($-(+)L_\Sigma^{Fe}'$) with an average magnitude of of .017 (.018), or 2.8\% (2.8\%) when normalized by climatologic values ($L_\Sigma^{Fe}''$ ). When considering only large, deep mixing eddies,  89\% (92\%) of cyclones (anticyclones) exhibit $-(+)L_\Sigma^{I_{PAR}}'$ with an average magnitude of .022 (.022), or 2.8\% (3.0\%) when normalized by climatologic values.


The ultimate effect on $\mu'_\Sigma$ is driven by the relative contribution of both limitation terms, which are often modified in opposing directions. The dominance of respective limitation terms appears to vary seasonal. During the summer,  $\mu'_\Sigma$ is much better correlated with $(L_\Sigma^{Fe}')$ ($r=.66$ in cyclones; $r=.73$ in anticyclones) than $L_\Sigma^{I_{PAR}}'$ ($r=-.07$ in cyclones $r=-.17$ in anticyclones). Alternatively, during the winter  $\mu'_\Sigma$ is much better correlated with $L_\Sigma^{I_{PAR}}'$ ($r=.63$ in cyclones $r=.64$ in anticyclones) than $L_\Sigma^{Fe}'$ ($r=.06$ in cyclones $r=.13$ in anticyclones). The mechanisms behind this seasonal variability are discussed in \textbf{Sec. 4.3}.





%%%%%%%%%%%%%%%%%%%%%%%%%%%%%%%%%%%%%%%%%%%%%%%%%%%%
                %% 4. Analysis  %%
%%%%%%%%%%%%%%%%%%%%%%%%%%%%%%%%%%%%%%%%%%%%%%%%%%%%
\section{Analysis}

\textcite{SongSeasonalvariationcorrelation2018} have shown that the seasonal variability in the correlation between $[Chl]'_S$ and $SSH'$ in this simulation agrees with observations, justifying a deeper investigation into the simulated mechanisms by which eddies modify Southern Ocean phytoplankton populations. Here we analyze the step-by-step response of depth-integrated populations to eddies to provide a greater statistical context for how Southern Ocean eddies modify divisions rates, and in doing so help test the hypothesis proposed by \textcite{SongSeasonalvariationcorrelation2018} that anticyclones (cyclones) enhance (suppress) division rates in the summer, and suppress (enhance) them in the winter.

Unlike the majority of correlative work which treats each grid cell independently as cyclonic or anticyclonic (i.e. \textcite{GaubeRegionalvariationsinfluence2014}, \textcite{SongSeasonalvariationcorrelation2018}) our statistics specifically operate on coherent mesoscale structures. Using closed contours in the $SSH'$ field to isolate eddies reduces the probability of including spurious, non-rotating mesoscale features and helps isolate the mechanisms that are unique to actual eddies. \textcite{SongSeasonalvariationcorrelation2018}, for instance, estimate that 33\% of analyzed data are not enclosed by an $SSH'$ contour at all. Still, the \textcite{Faghmousdailyglobalmesoscale2015} methodology we employ is apt to misidentify meanders as eddies. We experimented with several methods to isolate ``true'' eddies, including filtering for size, shape, and propagation distance, in addition to a constraint requiring eddies to exhibit closed contours in both the $SSH'$ and $SSH$ fields. Ultimately, no filter dramatically altered the qualitative nature of our results and we deemed the most robust definition was the simplest. 
    
Further, we consider the depth-integrated biological response to eddy perturbation, rather than focusing in surface concentrations (as done in \textcite{Misumiironbudgetocean2014,GaubeRegionalvariationsinfluence2014,FrengerImprintSouthernOcean2018}, etc.) which are prone to overlook potentially large chlorophyll anomalies that have been observed below the surface of eddies \parencite{SiegelMesoscaleeddiessatellite1999,McGillicuddyEddywindinteractions2007} and discount the importance of dilution during deep mixing \parencite{BehrenfeldAnnualcyclesecological2013}. By employing community means (see \textbf{Sec. 2.3}) our results account variability in the distribution of the biomass profile and community composition. 

We find that simulated depth-integrated phytoplankton populations respond to Southern Ocean cyclones and anticyclones with a striking symmetry and consistently inverse statistical response. To better understand the mechanics of this response, here we compare simulated, observed and theoretical estimates of $MLD'$ (\textbf{4.1}), quantify the relative contribution of anomalous iron sources (\textbf{4.2}), and address the seasonal decoupling of light and iron limitation terms (\textbf{4.3}).

%%%%%%%%%%%%%%%%%%%%%%%%%%
            %% 4.1 Differences in theoretical, simulated and observed mixed layer modification  %%
%%%%%%%%%%%%%%%%%%%%%%%%%%
    
\subsection{Comparison of theoretical, simulated and observed $MLD'$}

Theoretically, cyclones(anticyclones) should preferentially induce anomalously shallow(deep) $MLD'$, however, this is not apparent in the overall distribution of eddies. This is largely because small eddies ($L_S<50 km, \ Amp.<10 cm$) passing through low energy regions with shallow background mixing ($\overline{MLD}_{Clim}<100m$) preferentially exhibit ``backwards'' $MLD'$ ($+(-) MLD'$ in cyclones (anticyclones)). From a numerical perspective, the resolution of the model integration is laterally $\sim10km$ and vertically 10m, thus smaller eddies with shallower mixed layers are not resolved as well as larger eddies. From a physical perspective, the magnitude of $MLD'$ in these less energetic eddies is consistently quite small, which agrees with observations of low magnitude anomalies in quiescent Southern Ocean eddies \parencite{HausmannObservedmesoscaleeddy2017} and the theoretical underpinning that smaller eddies should not dramatically deform isopycnals. 

Geographically, eddies under heavily ice covered waters ($\overline{Ice}_{Clim}\>.8$) exhibit the most consistently ``backwards'' $MLD’$ (see \textfbf{Fig. 7b}. Note, however, that heavy ice coverage is apt to damp eddy amplitudes [SOURCE??] and protect the sea surface from wind shear need to drive deep mixing [SOURCE??]. In the simulation, eddies below heavy ice exhibit a mean amplitude of only $2.2cm$ and mean background mixing of only 69m. In turn 75\% (77\%) of cyclones (anticyclones) display $+(-) MLD'$ with an average deformation of 14m (-15m). Despite the relatively small sample (13451 realizations from 1400 tracks), the difference in distributions is statistically significant at the 99.9\% confidence level. Even when considering the combined effect of worse resolution and weaker physical forcing, it is not entirely clear why this response is systematically backwards and not simply mixed. 

Larger eddies ($L_S>50 km, \ Amp.>10 cm$) with deep background mixing ($\overline{MLD}_{Clim}>100m$) do, however, largely agree with theory \parencite{McGillicuddyMechanismsPhysicalBiologicalBiogeochemicalInteraction2016} and observations \parencite{HausmannObservedmesoscaleeddy2017}. Peak $MLD'$ is simulated and observed \parencite{HausmannObservedmesoscaleeddy2017} to occur in the late winter along highly energetic current systems and increase with eddy amplitude. The magnitude of simulated anomalies reported in \textbf{Fig. 4, 7 \& Tab. 1} is smaller than observed, however this is because we average over the entire lateral eddy profile rather than just the core where anomalies are maximized \parencite{HausmannObservedmesoscaleeddy2017}. If we isolate the core, defined as a circle with a radius half the size of the eddy radius, the magnitude of simulated anomalies also increases. Peak anomalies in the core of simulated large, deep mixing, late-winter anticyclones are on average $59m$ and compare very favorably to the $60m$ anomalies observed by \textcite{HausmannObservedmesoscaleeddy2017} in late-winter, energetic SO regions (e.g. the ACC, and western boundary currents). The same subset of simulated cyclones, however, exhibits $MLD' =-63m$, rather than the less extreme $-30m$ found in the observations.  It unclear why the model does not reproduce this asymmetry in the magnitude of $MLD'$, but unlikely that the qualitative nature of light and iron modification in cyclones is fundamentally biased.  
    
   
%%%%%%%%%%%%%%%%%%%%%%%%%%
            %% 4.1 Relative contribution from iron transport pathways  %%
%%%%%%%%%%%%%%%%%%%%%%%%%%
    
\subsection{Relative contribution of iron transport pathways}
    
Community mean iron concentrations are consistently skewed to favor $-(+)[Fe]_\Sigma$ in cyclones (anticyclones). The source of the simulated  anomaly is, however, complicated by the existence of several pathways that could theoretically account for it. In \textbf{Fig. 8} we plot the distribution of the cumulative anomalous contribution from all four source/sink terms considered in \textbf{Sec. 3.3.2} and compare them to the ensuing iron anomaly that has developed over the course of the eddy.  The cumulative community mean vertical advection tendency ($\frac{d[Fe]}{dt}_{\Sigma, W, cum}$), mixing tendency ($\frac{d[Fe]}{dt}_{\Sigma, M, cum}$), and biological tendency ($\frac{d[Fe]}{dt}_{\Sigma, J, cum}$) are all integrated cumulatively across each eddy's lifetime to quantify the total anomalous iron supplied to the average phytoplankton by a given mechanism. The $Adv. 
\ Potential$ term does not represent the actual horizontal transport but rather the anomaly that would emerge solely by nonlinear propagation across the climatologic gradient. It can be considered the upper bound for the bulk contribution of all lateral advection processes. Statistically, the supply of iron via all physical sources is depressed (enhanced) in cyclones (anticyclones) leading to  $-(+)[Fe]_\Sigma$ and reduced (increased) biological consumption. The largest source of new nutrients, is vertical advection, which appears to be dominated by Ekman Transport. 

Vertically, the predominate direction of $[Fe]'_\Sigma$ and $\frac{d[Fe]}{dt}_{\Sigma, W}$ unambiguously preclude eddy pumping from dominating the vertical supply of iron to/from simulated eddies. Historically, observations have attributed enhanced (depressed) chlorophyll at the core of cyclones (anticyclones) to eddy pumping \parencite{FalkowskiRoleeddypumping1991,SiegelMesoscaleeddiessatellite1999, McGillicuddyInfluencemesoscaleeddies1998}, but more recent work has identified substantial regional variability in it's importance \parencite{GaubeRegionalvariationsinfluence2014, GaubeSatelliteObservationsMesoscale2014} and highlighted regions, such as the Gulf Stream \parencite{GaubeinfluenceGulfStream2017, ZhangImpactsMesoscaleEddies2018}, and in-situ case studies \textcite{McGillicuddyEddywindinteractions2007} where Ekman pumping appears to dominate. 
    
The simulated vertical velocities associated with Ekman pumping in the Southern Ocean are on average -17 (15) cm/d in cyclones (anticylones), and exceed -70(65) cm/d in the strongest 5 percent of cyclones (anticyclones). This is in line with a globally observed mean of roughly 10 cm/d \parencite{GaubeSatelliteObservationsMesoscale2014} and upper bound on the order of 100 cm/d \parencite{MartinMechanismsverticalnutrient2001}. Further, simulated anomalous Ekman velocities operate in the same direction and on the same order of magnitude as the average total vertical velocity anomaly ($-??cm/d$ in cyclones; $??cm/d$ in anticyclones), emphasizing the contribution of Ekman pumping to variability in vertical advection. As the simulated magntidue of Ekman pumpiung increases, so too doe the associated anomalous vertical flux of iron (\textbf{Fig. 8g,h}) Observed, vertical velocities induced by eddy-pumping in the open ocean typically do not exceed 100 $cm/d$ \parencite{SiegelMesoscaleeddiessatellite1999,GaubeSatelliteObservationsMesoscale2014} and quickly subside after formation, so it is not surprising that Ekman pumping can  dominate anomalous vertical advection, particularly in the Southern Ocean where wind speeds are quite strong. 
    
More surprising is the relative contribution from the mixing flux. Statistically, the cumulative community mean contribution of iron from anomalous mixing ($\frac{d[Fe]}{dt}_{\Sigma, M, cum}$) is in the expected direction (-(+) in cyclones(anticyclones) but 8.5 (10.3) times smaller than the advective contribution ($\frac{d[Fe]}{dt}_{\Sigma, W, cum}$) in cyclones (anticyclones). Further $[Fe]'_\Sigma$ appears more sensitive to polarity than to $MLD'$. For instance, while 71\% (73\%) of all anticyclones exhibit $+(-)[Fe]'_\Sigma$, only 51\% (51\%) of all eddies with $+(-)MLD'$ exhibit $+(-)[Fe]'_\Sigma$. In fact, 73\% (75\%) of cyclones (anticyclones) with anomalously deep (shallow) mixed layers still exhibit $-(+)[Fe]'_\Sigma$.

The heightened role of Ekman pumping relative to mixed layer modification agrees with work by \textcite{ZhangImpactsMesoscaleEddies2018}, who conclude that the vertical supply of nitrate in Gulf Stream eddies is dominated by Ekman pumping rather than changes vertical turbulent diffusion, but contradicts \textcite{SongSeasonalvariationcorrelation2018} who conclude mixing is the dominant driver of iron into/out of eddies. \textcite{SongSeasonalvariationcorrelation2018} use an identical model integration to ours but compute the iron budget over the entire mixed layer and do not consider the distribution of the biomass profile. This is an important distinction because, as seen in \textbf{Fig. 7a, b}, the anomalous mixing flux is much stronger near to the base of the mixed layer, but biomass is more likely to be congregated near the surface. Thus when we compute comunity mean tendencies, much less iron is supplied by mixing to the bulk of the biomass population than that which is injected into the base of the mixed layer, particularly if deep mixing anomalies are transient and actitucker
ve mixing is not sustained. We do, however, see that $[Fe]'_\Sigma$ increases (decreases) in cyclones (anticyclones) with the magnitude of $MLD'$. For instance, anticyclones with meaningful deep mixing anomalies ($MLD'>10m$) exhibit $[Fe]'_\Sigma$ with a mean ($.021 \mu mol/m^3$ ) 350 \% times higher those with shallower mixing anomalies ($MLD'<10m$).  $[Fe]'_\Sigma$ further increases to $.056 \mu mol/m^3$ in anticyclones with very deep mixing anomalies ($MLD'>75m$). This suggests that once iron is entrained from depth by anomalous mixing in anticyclones it can be transported by Ekman pumping even if active mixing subsides. On the other hand, even if cyclones mix deeper Ekman downwelling suppresses the vertical transport of iron to surface populations. Note, also, that 33\% of the \textcite{SongSeasonalvariationcorrelation2018} data is not actually within a coherent eddy structure, and not likely to experience systematic Ekman pumping driven by coherent, rotating, surface currents, and thus may dilute the role of vertical advection in their analysis. Our results, however, should not diminish the importance of mixed layer modification, but rather emphasize the importance of Ekman pumping working in conjunction with mixing. 

We focus on vertical nutrient pathways because these can supply/suppress new nutrients capable of stimulating/stifling new production. When lateral processes generate an anomaly, it can often be attributed to a spatial redistribution of biomass and productivity rather than a disruption to net new production on a regional scale \parencite{FrengerImprintSouthernOcean2018}. It is difficult to budget exactly what is or is not new iron but if we are to believe eddies can actually modify new production on a global basis then it is important to ensure that the scale of vertical transport is not dramatically smaller than that of lateral transport. The potential for the lateral transport of iron across the climatologic gradient to produce an anomaly ($Adv. \  Potential$) is slightly higher than the total anomalous iron supplied by vertical transport (See \textbf{Fig. 9}), however, the $Adv. \  Potential$ represents the theoretical ceiling from lateral advection and not the actual flux. Further, the cumulative vertical contribution is similar is size and direction to the corresponding $[Fe]'_\Sigma$, confirming that vertical transport anomalies act on a relevant scale. Altogether, the direction of $[Fe]'_\Sigma$ is aligned with the direction of total anomalous vertical iron transport ($\frac{d[Fe]}{dt}_{\Sigma, W, cum}$ + $\frac{d[Fe]}{dt}_{\Sigma, M, cum}$) in almost the same number of eddies (62\%) as when it is aligned with the direction of the $Adv. \  Potential$ (66\%) suggesting that both vertical and lateral processes operate on similar scales without either ubiquitously dominating iron transport. Note that the contribution of iron from lateral sources does not compromise or counteract vertical transport, but does caution against assuming the entire observed iron anomaly is actually ‘new’ in a net regional context. 



%%%%%%%%%%%%%%%%%%%%%%%%%%%%%%%%%%%%%%%%%%%%%%%%%%%%%%%%%%%%%%%%%%%%%%%%%%%%%%%%%%%%%
%%  Division Rates and the seasonal decoupling of light and iron limitation terms
%%%%%%%%%%%%%%%%%%%%%%%%%%%%%%%%%%%%%%%%%%%%%%%%%%%%%%%%%%%%%%%%%%%%%%%%%%%%%%%%%%%%%
    
\subsection{Division Rates and the seasonal decoupling of light and iron limitation terms}
    
The ultimate effect of eddies on $\mu_\Sigma'$ is predominately driven by their relative influence on iron and light limitation. $(L_\Sigma^{Fe}')$ and $(L_\Sigma^{I_{PAR}})'$ are not, however, independent. Moreover, the balance between the often competing effects of light and iron limitation appears to vary seasonal. To help understand the variability in their relative contribution to $\mu_\Sigma'$ \textbf{Fig. 10} plots the iron limitation anomaly as a function of the coincident light limitation anomaly for each season. 

We first note that $L_\Sigma^{I_{PAR}}'$ is predominately driven by mixing and can be thought of as a proxy for $MLD'$, while $L_\Sigma^{Fe}'$ can be modified by several sources of iron. If mixing was the only meaningful mechanism affecting iron supply then we would expect $L_\Sigma^{I_{PAR}}'$ to be consistently anti-correlated with $L_\Sigma^{Fe}'$. This, however, contradicts the relatively flat slopes seen in Quadrants I and III, where $L_\Sigma^{Fe}'$ does not appear to vary with $L_\Sigma^{I_{PAR}}'$. Here, it seems that Ekman upwelling keeps iron elevated ($L_\Sigma^{Fe}'>0$) in anticyclones even as the $MLD$ shoals and light limitation is relieved (Quadrant I) but Ekman down welling in cyclones keeps iron depressed ($L_\Sigma^{Fe}'<0$) even as the MLD deepens and light limitation is exacerbated (Quadrant III).

In Quadrants II and IV, where Ekman pumping and mixing should transport iron in the same direction, slopes are negative but vary substantially with season. In the winter (solid lines) slopes are much steeper than in the Summer (dotted line) indicating that the ``cost'' associated with relieved iron stress (more positive) in terms of coincident light limitation is much harsher (more negative). This agrees with \textcite{SongSeasonalvariationcorrelation2018} who hypothesize that during the summer, anticyclones can relieve substantial iron stress without inflicting too much light stress, by deepening shallow mixed layers without leaving the euphotic zone, but during the Winter, background mixing is already well below the euphotic depth and iron concentrations are higher, meaning that deep mixing can only relieve iron stress at a much greater ``cost''.

This theory is further supported by the relationship between $\mu'_\Sigma$ and $\overline{MLD}_{clim}$ seen in \textbf{Fig 7k}. Initially, $\mu'_\Sigma$  increases with increasing $\overline{MLD}_{clim}$ but quickly reaches an inflection point and sharply drops off, eventually flipping sign once the $\overline{MLD}_{clim}$ exceeds about 150m. Geographically, this sign flip coincides with the intrusion of flipped division rate anomalies into the deep mixing winter ACC region (\textbf{Fig. 7b, c}). 

All together, seasonal variability in $\mu_\Sigma'$ seems to support the hypothesis developed by \textcite{SongSeasonalvariationcorrelation2018} and \textcite{FrengerImprintSouthernOcean2018} but further investigation into the simultaneous effect of dilution and grazing are required to be confident that variability in depth-integrated division rates is driving variability in surface chlorophyll concentrations. $[Chl]_S$ is not not necessarily well coupled to $\mu_\Sigma$ \parencite{RohrVariabilitymechanismscontrolling2017} and depressed $[Chl]_S$ could additionally be explained by dilution across a deeper mixed layer or increasing grazing rates \parencite{BehrenfeldAnnualcyclesecological2013}. 

